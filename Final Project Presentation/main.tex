\documentclass{beamer}

\usepackage[utf8]{inputenc}
\usepackage{amsmath}
\usepackage{mathtools}
\DeclareMathOperator*{\argmax}{argmax} % thin space, limits underneath in displays
\mode<presentation>{
\usetheme{Madrid}
\usecolortheme{TOSU}
}

% Load biblatex package and .bib document
\usepackage{natbib}
\bibliographystyle{unsrtnat}

\AtBeginSection[]{
  \begin{frame}
  \vfill
  \centering
  \begin{beamercolorbox}[sep=8pt,center,shadow=true,rounded=true]{title}
    \usebeamerfont{title}\insertsectionhead\par%
  \end{beamercolorbox}
  \vfill
  \end{frame}
}

%Information to be included in the title page:
\title{News Article Classification}
\subtitle{(Multi-Label Learning: ML-KNN \& BP-MLL)}
\author[STAT 6500]{Lauren Contard, Archit Datar, Bobby Lumpkin, Haihang Wu}
\institute[OSU] % Your institution as it will appear on the bottom of every slide, may be shorthand to save space
{
The Ohio State University \\ % Your institution for the title page
\medskip
STAT 6500 % Your email address
}
\date{}



\begin{document}

\frame{\titlepage}

\begin{frame}
\frametitle{Overview} % Table of contents slide, comment this block out to remove it
\tableofcontents % Throughout your presentation, if you choose to use \section{} and \subsection{} commands, these will automatically be printed on this slide as an overview of your presentation
\end{frame}

\section{Introduction and Problem Statement}

\section{KNN Based Approaches}

\subsection{Binary Relevance}

\subsection{ML-KNN Algorithm}

\subsection{Results}

\section{Neural Network Based Approaches}

\subsection{Architectures: Feed Forward \& Recurrent Networks}

\subsection{Loss Functions: Cross Entropy vs BPMLL}

\subsection{Results}

\section{Discussion and Conclusions}

\begin{frame}[t]{References}
    \nocite{mlknn}
    \nocite{bpmll}
    \bibliography{mll_osu}
\end{frame}

\end{document}

